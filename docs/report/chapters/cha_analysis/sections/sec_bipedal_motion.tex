%!TEX root= ../../../report.tex

\section{Bipedal locomotion} % (fold)
\label{sec:bipedal_walking_and_running_gaits}
This section contains a superficial comparative study of the characteristics of human-like walking and running patterns.
The reason is to help justify in the following chapters the decisions taken during the design.

Although walking and running motion patterns in humans might resemble each other at first glance, they contain important variations that imply that a robot able to walk at a certain speed is not necessary capable of running at the same speed, or running at all.
The existence of the so-called "flight phase" in running but not in walking cycles is the main difference between them.
This phase encompasses the period in which there is no contact between the feet and the ground and causes the main variations in energy and trajectory generation.

\subsection{Walking and running gaits comparison} % (fold)
\label{sub:walk_and_run_comparison}
The differences in both joint kinematics and kinetics between both patterns arisen from this distinction are studied in \cite{grimmer}, Manuscript I, for a wide range of speeds.
From the cited paper it can be seen that the joint positions for a whole cycle in both walking and running patterns are very similar, specially for hip and knee.
Furthermore, the joint velocity profiles have almost the same shape but differ in the magnitudes, being higher the angular speeds reached for running.
Similar conclusions can be extracted from a quick comparison between the torque and power profiles in a running and walking cycle.
While the shapes of the functions for the three joints are the really close, with bigger differences for the ankle, the values of the curve are in average smaller for human walking than for running.
As expected, the conclusion is that higher requirements for joint speed, torque and power are expected for running, specially for the knees and the ankles.

% Impact forces comparison here
Besides, the foot strike and impact forces during the landing phase in both cases change, both in intensity and profile. ***

% subsection walk_and_run_comparison (end)

% section bipedal_walking_and_running_gaits (end)