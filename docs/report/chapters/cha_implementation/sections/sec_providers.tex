%!TEX root = ../../../report.tex
\section{Providers} % (fold)
\label{sec:providers}
As explained in the management of the project section \ref{sec:project_management}, all the components that could not be manufactured or were not available at the university facilities had to be ordered.
The providers have been selected among the range offered by the Mærsk Mc-Kinney Møller Institute.
The main selection criteria of utilized external components has been standardization.
Following the underlying principles of this project, only low-cost, standard pieces have been acquired. As an example, RuBi only needs three types of screws (only M3 metric) and one kind of bearing.
A list with all the components bought and its providers is shown in the Table \ref{tab:material_cost}. 

\subsection{Components delivery} % (fold)
\label{sub:components_delivery}
In the initial scheduling of the project, conducted at its very beginning, it was planned to have defined and ordered by the end of April all the necessary components that had to be bought.
This was decided assuming that the delivery of the parts could take up to three weeks in a reasonable scenario.
The acquisition of external components was considered during the planning of the project as a strong time constraint, and the rest of the workflow was scheduled to adapt to this step.
However, despite the fact that most of the parts were received within the first few week after ordering, the springs suffered a severe delay that prevented their scheduled assembly and produced a necessary reschedule of the project flow in order to adapt to this.
This event has mainly impeded carrying out the devised experiments presented in \ref{cha:experiments} and some basic performance tests that were considered essential.
This is further discussed in \ref{cha:results} and \ref{cha:discussion}.

% subsection components_delivery (end)

% section providers (end)