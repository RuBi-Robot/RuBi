%!TEX root = ../../../report.tex
\section{Machining} % (fold)
\label{sec:machining}
Along with 3D printing, other parts have needed to be machined.
Though its allows the creating of complex geometries, the 3D printers used have a limited printing volume.
Additionally, the mechanical properties given by the PLA do not satisfy all the needs of the used parts.
The rods of the axis, the beams used to create the structure, the carbon fiber tubes are some examples of this requirements of size and mechanical stress.


This materials usually come in a raw format that must be then modified in order to get the desired component.
As an example, the rods came in bars of one meter that have been cut and filed according the design.
Other example are the carbon fiber tubes.
These have been designed to carry the wiring inside them.
Thus some holes have been performed in the bottom and top part of the tube in order to insert them.
Knowing that this kind of operations is human error prone, non-critical parts have been machined or no such it can affect to the correct behavior of the device.
Furthermore, the actuation has always tried to be as much correct as possible and always following the appropriate security rules.
The drawings for such parts are included as appendices in \ref{app:mechanical_drawings}.
% section machining (end)