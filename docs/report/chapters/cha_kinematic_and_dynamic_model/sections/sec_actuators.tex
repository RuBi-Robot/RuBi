%!TEX root= ../../../report.tex

\section{Actuators}
\label{sec_actuators}
The previous calculations in this chapter have been carried out primarily in order to define the load requirements of the application the actuators will be used for.
In the present case, as explained, the application consists in walking and running gaits generation for the robot Rubi at a wide, not-fully-determined range of speeds. 
Due to the complexity of modeling such a wide task requirements, it was decided to study the case of the robot hopping on one leg assuming that the drive requirements for walking/running would be accomplished by actuators able to perform this action.

Considerations about the mechanical transformations of the motor output power before being driven to the load such as transmission mechanisms are not detailed here, but in \ref{sub:pulleys_and_belts}.
The present explanations are for the motor shaft and gearbox selection.

Generally, the process of selection of electric motors entails answering the following queries about the load and the available power supply:

\begin{itemize}
\label{list:motor_selection}
	\item Torques applied by the load to the motor shaft
	\item Accelerations involved
	\item Inertias of the masses 
	\item Maximum voltage and current available 
	\item Type of operation (continuous, intermittent, reversing)
\end{itemize}

The first three points in \ref{list:motor_selection} have been answered in the previous pages of this chapter.
Due to the intermittent operation nature of the application, and aiming at constructing as further work an on-board electronics set, batteries have been selected to supply the necessary power to the motors.
