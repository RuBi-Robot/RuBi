%!TEX root= ../../../report.tex

\section{Springs influence in the actuators}
\label{sec_springs}
This section deals with the calculations carried out in order to model the influence of the implemented compliance in terms of energy efficiency and joint torque requirements.
As explained in \ref{sec:joints}, the selection of the elastic actuators installed and their configuration on RuBi has been left as a customizable feature for research purposes.
This, together with the fact that RuBi has not been designed to optimize a specific type of gait made pointless the calculation of optimal values of springs.
Thus, the following is just a theoretical frame presented to provide an easy method to compute the influence of the springs in the performance once the user has selected them.
It was meant to be used for the experiments devised on compliance influence in hopping motion, presented in \ref{cha:experiments}.

\subsection{Torque and energy contribution of passive actuators} % (fold)
\label{sub:torque_contribution_of_passive_actuators}
All the springs that can be implemented in RuBi are torsional, as shown in \ref{sub:spring_integration}, whose general equations for torque and energy storage are shown in \ref{eq:torsion_spring}. 

\begin{equation}
\label{eq:torsion_spring}
\begin{aligned}
	\tau_{S} &= -K \Delta \theta \\
	U_{S} &= \frac{1}{2}K \Delta \theta^2
\end{aligned}
\end{equation}

The contribution of the springs torques to the system actuation is therefore modeled introducing the torque equation in \ref{eq:torsion_spring} on the right side of \ref{eq:dynamics_eq1}, which would yield equation \ref{eq:dynamics_eq2}.
Where the torques vector has been divided into the motors and springs inputs.

\begin{equation}
	\label{eq:dynamics_eq2}
	B(q)\ddot{q} + C(q,\dot{q})\dot{q} + g(q) = \tau_{M} + \tau_{S}
\end{equation}

The values of $\Delta \theta_{i}$ can be obtained through forward kinematics for given trajectories, while $\tau_{i}$ can be computed from equation \ref{eq:dynamics_eq1} given the ground contact force model, which is the only external force applied to the system.
Thus, the first formula in \ref{eq:torsion_spring} can be used to calculate the optimal $K$ for a desired joint trajectory and force model, for instance.
Knowing $k_{i}$ and $\Delta \theta_{i}$, the energy stored per gait cycle by each spring can be computed and utilized for studying their influence on the performance of the robot.

\subsection{Joint kinetics} % (fold)
\label{sec:joint_kinetics}
The motor power requirements for each joint can be calculated through equation \ref{eq:motor_power} for direct drive transmission.
This formula does not include inertias, frictions or any other efficiency coefficients. 

\begin{equation}
\label{eq:motor_power}
	P_{m} = \dot{\theta}_{m} \tau_{m}
\end{equation}

By definition of \ref{eq:motor_power}, the value of $P_{m}$ can be positive (motor thrusting) or negative (motor dumping).
Thus, the peak power in the motion cycle can be computed as the its maximum absolute value.
Besides, the equation of motor power is used in \ref{eq:energy_requirement} for calculating the overall energy requirements for a whole motion cycle, employing only absolute values as well.

\begin{equation}
\label{eq:energy_requirement}
 	E_{cycle} = \int{P_{m+}(t) dt} + \int{P_{m-}(t) dt}
 \end{equation} 

\subsubsection{Influence of elastic actuation} % (fold)
\label{sub:influence_of_elastic_actuation}
The above presented are general formulations.
However, the introduction of elastic transmission between motor and load through springs modify the computation of motor power.
The changes resulting from implementing the configurations shown in Figure \ref{fig:sea}, \ref{fig:pea} and \ref{fig:sea_pea} yield equations \ref{eq:SEA_power}, \ref{eq:PEA_power} and \ref{eq:SEA_PEA_power} respectively, as per \cite{grimmer}.
For a complete and detailed description of the parameters in these equations, the reader is referred to the cited paper.

\begin{equation}
\label{eq:SEA_power}
	P_{m} = \tau_{m} \left(\dot{\theta}_{t} + \frac{\dot{M}_{ex}}{K_{s}}\right)
\end{equation}

\begin{equation}
\label{eq:PEA_power}
	P_{m} = (F_{ex} + (K_{p} \Delta \theta_{t})) \dot{\theta}_{t}
\end{equation}

\begin{equation}
\label{eq:SEA_PEA_power}
	P_{m} = \left(F_{ex} + (K_{p} \Delta \theta_{m}) \left(\theta_{t} + \frac{\dot{F}_{ex}}{K_{s}}\right)\right)
\end{equation}

% subsubsection influence_of_elastic_actuation (end)

% section joint_kinetics (end)