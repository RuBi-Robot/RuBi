%!TEX root = ../../report.tex
\chapter{Discussion} % (fold)
\label{cha:discussion}
% a reference to the main purpose of the study
The purpose of the project was the construction of a robust, low-cost
and reconfigurable bipedal platform for human-like walking and running locomotion
generation and control studies.
Once the results of the process are collected, they deserve an evaluation in terms of the goals aimed with the thesis and the development of events withing the past months.

\section{The first prototype of RuBi} % (fold)
\label{sec:the_prototype_of_rubi}
The core of the devised framework for locomotion studies was the biped robot RuBi.
The design and construction of the first prototype has been fully achieved as planned despite the eventualities faced.
Its theoretical sensory/motor and mechanics features accomplish most of the targeted requirements for a robot of its characteristics, inspired in similar, existing models in the literature and the authors own intuition.
However, the next chapter details some tasks left as further work considered necessary for a complete success on the implementation of the fully functional prototype has conceived.

The main deficiency on the development process remains the absence of empirical data to corroborate the accuracy of the theoretical calculations carried out.
This has been due, as already explained, to the ambitious scope of this master's thesis, the delay in the reception of some fundamental parts of the robot or the unforeseen eventualities that a project of these characteristics always implicates.
Notwithstanding this fact, that could be firstly mistaken for a result of a misled planning of the project, actually highlights one of its most remarkable facts.
This is, that the achievement of the intended goals to their current extent has been carried out in a notable time lapse of 4 months.
% section the_prototype_of_rubi (end)

\section{The simulation and control architectures} % (fold)
\label{sec:the_simulation_and_control_architectures}
The goal of a more general simulation environment than the existing LPZ robots used with the DACbot has been accomplished through the implementation of a Gazebo+ROS Control structure.
The system stands out for facilitating the implementation of locomotion controllers and the possibility of testing them in the simulation and in the real robot.
Two have been created and tested in simulation, although not in the real setup.
The simulation has proved to be a good tool for qualitative analysis but not for tests assuming real-world conditions, since that still needs further development.
% section the_simulation_and_control_architectures (end)

\section{The mathematical framework} % (fold)
\label{sec:the_mathematical_framework}
The mathematical framework constructed has been utilized to simulate the dynamics of the robot and model the physics of a vertical jump on one leg.
Its application has led the motor+gearbox selection process.
Its use has also provided for a theoretical background to lead the design of the geometrical and inertial parameters and justify the criteria that has led the mechatronics design of the robot.
Thus, the advantages of reducing the moments of the inertia of the limbs, rearranging their masses or finding the right proportion of their lengths has been provable and quantifiable thanks to this set of equations.

% section the_mathematical_framework (end)

\section{The test bench for experimentation} % (fold)
\label{sec:the_test_bench_2}
The devised test bench whose goal was the constraint of the robot movements to its sagital plane has been constructed and installed on top of the existing treadmill.
However, its functionality has not been tested further than manually because of the problems with the robot prototype described above.
% section the_test_bench (end)
% chapter discussion (end)