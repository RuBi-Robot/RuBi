%!TEX root = ../../../report.tex
\section{Goals}
\label{sec:goals}
The objectives of the research and mechatronics work this thesis has implied are detailed below.
Due to the ambitious scope of the project and the wide range of tasks involved it was hard to assess at the initial proposal the time needed and therefore the main goals of the thesis. 

\subsubsection{The Running Bipedal robot RuBi} % (fold)
\label{ssub:the_running_bipedal_robot_rubi}
The robot Rubi has been devised as a platform to be used for testing new control algorithms used for the generation of human-like walking and running gaits and the transition between them.
Therefore, it has to count on a reliable hardware structure able to offer all the needed functionalities, together with a robust software frame capable of dealing with the requirements of such algorithms.
The capabilities that walking and running gaits generation requires are detailed in the next chapters.
% subsubsection the_running_bipedal_robot_rubi (end)


\subsubsection{RuBi as a study framework} % (fold)
\label{ssub:rubi_as_a_study_framework}
This thesis aimed at creating not only the robot described below, but a whole framework for legged locomotion studies.
To do so, some more tools besides a first prototype of the robot were targeted as goals, and are listed in \ref{list:goals}.
Some of them have emerged from the actual process of design and construction of the robot, and others have been included with the view on its future use for research.

\begin{itemize}
\label{list:goals}
\item A more general simulation environment than the existing one, based on ROS and Gazebo and eliminating dependencies with LPZ robots, together with a fully-scalable simulation model of the robot. Furthermore, the implementation of the ROS Control set of packages \cite{ros_control} in the system allows for a simple implementation of controllers that can be easily interfaces with Gazebo or the actual robot. This has been tested with two custom controllers.
\item A test bench designed to hold the robot limiting its movements to the motion plane under study and guaranteeing its stability over a treadmill for locomotion experimentation. 
\item A mathematical framework to be used for computation of both kinematic and dynamic parameters and potentially for developing classical controllers from the dynamics model of the robot.
\item An easy to modify and extend software interface with the electronic hardware, aimed at providing a simple and functional control of the robot.
\end{itemize}

Even thought some of these goals have not been fully accomplished in the sense that they still need some improvement to be fully functional, as explained before all the necessary steps to ease the continuity of their development have been taken.
Furthermore, in the original description of the project was also stated that, if possible, some experiments would be devised and carried out in order to test the functionality and physical properties of the prototype. 
However, some eventualities detailed in the next chapters have prevented the project from reaching that stage and a more complex experimentation phase besides the simple tests of functionality are left as further work.
% subsubsection rubi_as_a_study_framework (end)
