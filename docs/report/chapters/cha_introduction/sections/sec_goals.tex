%!TEX root = ../../../report.tex
\section{Goals}
\label{sec:goals}
The objectives that have led the research and mechatronics work this thesis has comprised are detailed below.
Due to the ambitious scope of the project and the wide range of tasks involved, it was hard to assess at the initial proposal the time requirements and therefore the main goals of the thesis. 

\subsubsection{The Running Bipedal robot RuBi} % (fold)
\label{ssub:the_running_bipedal_robot_rubi}
The robot RuBi has been devised as a platform to be used for testing new control algorithms used for the generation of human-like walking and running motion and the transition between them.
Therefore, it has to count on a reliable hardware structure and actuation system able to offer all the needed functionalities, together with a robust software frame capable of dealing with the requirements of such algorithms\footnote{The capabilities that walking and running gaits generation requires are detailed in \ref{sec:bipedal_walking_and_running_gaits}.}.
% subsubsection the_running_bipedal_robot_rubi (end)


\subsubsection{RuBi as a study framework} % (fold)
\label{ssub:rubi_as_a_study_framework}
This thesis aimed at creating not only the robot described below, but a whole framework for legged locomotion studies.
To do so, some more tools besides a first prototype of the robot were targeted as goals, as they are listed in \ref{list:goals}.
Some of them have emerged from the actual process of design and construction of the robot, and others have been included with the view on its future use for research.

\begin{itemize}
\label{list:goals}
\item A more general simulation environment than the existing one, based on ROS and Gazebo and eliminating dependencies with LPZ robots \cite{lpzrobots}, together with a fully-scalable simulation model of the robot. Furthermore, a simple process of integration of new, custom controllers with the system is desired.
\item A test bench designed to hold the robot limiting its movements to its sagital plane and guaranteeing its stability over a treadmill for displacement experimentation. 
\item A mathematical framework to be used for computation of both kinematic and dynamic parameters and potentially for developing classical controllers from the dynamic model of the robot.
\item A powerful sensory/motor system and an easy to modify and extendable software interface with the electronic hardware, aimed at providing a simple and functional control of the robot.
\end{itemize}

Even thought some of these goals have not been fully accomplished in the sense that they still need some improvement to be fully functional, as explained before all the necessary steps to ease the continuity of their development have been taken.
Furthermore, in the original description of the project was also stated that, if possible, some experiments would be devised and carried out in order to test the functionality and physical properties of the prototype. 
% subsubsection rubi_as_a_study_framework (end)
