%!TEX root = ../../../report.tex
\section{Goals}
\label{sec:goals}
The objectives of the research and mechatronics work this thesis has implied are detailed below.
Due to the ambitious scope of the project and the wide range of tasks involved it was hard to assess at the beginning the time needed and therefore the main goals of the thesis. 
The framework for legged locomotion studies was decided to include some tools besides a first prototype of the robot:

\begin{itemize}
\item A more general simulation environment than the existing one, based on ROS and Gazebo and eliminating dependencies with LPZ robots, together with a fully-scalable simulation model of the robot. Furthermore, the implementation of the ROS Control set of packages \cite{ros_control} in the system allows for a simple implementation of controllers that can be easily interfaces with Gazebo or the actual robot. This has been tested with two custom controllers.
\item A test bench designed to hold the robot limiting its movements to the motion plane under study and guaranteeing its stability over a treadmill for locomotion experimentation. 
\item A mathematical framework to be used for computation of both kinematic and dynamic parameters and potentially for developing classical controllers from the dynamics model of the robot.
\item An easy to modify and extend interface with the on-board electronic hardware aimed at providing actuation and feedback to the robot.
\end{itemize}

Even thought some of these tasks have not been fully accomplished in the sense that they still need some improvement to be fully functional, as explained before all the necessary steps to ease the continuity of their development have been taken.
Furthermore, in the original description of the project was also stated that, if possible, some experiments would be devised and carried out in order to test the functionality and physical properties of the prototype. 
However, some eventualities detailed in the next chapters have prevented the project from reaching that stage and the more complex experimentation besides the simple tests of functionality are left as further work.