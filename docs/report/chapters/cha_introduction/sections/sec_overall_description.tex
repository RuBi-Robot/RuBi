%!TEX root = ../../../report.tex

\section{Overall description}
\label{sec:overall_description}
The present thesis aims at laying the foundations of the construction of a robust, low-cost and compliance-reconfigurable biped platform for human-like locomotion generation and control studies. 
The idea behind this project is to contribute to the on-going line of research in locomotion control at the AI department at the Maersk Mc-Kinney Møller Institute a new biped robot for further investigation.

The devise, construction and testing of a functional robot of this complexity is by definition a task of great magnitude that involves the conjunction of several disciplines within the robotics field, including electromechanics, structural design, materials science and computer science and control, among others.
Such a complex and extended work, together with the time constraints imposed, have made the authors of the project conceive its development with the view in a further growth beyond the scope of a single master thesis.
With this idea in mind, one of the goals aimed here has been to carry out this project facilitating to the farthest extent possible the continuity of its development by new students and researchers. 
Hence, both the hardware and software tools utilized and developed have been thought to be as general-purpose as possible, and all the necessary information required to replicate the work performed has been carefully documented here.

With the above introduced as the guidelines for the whole process, the mechanical design of the robot has been conducted striving for simplicity and functionality, for which 3D-print manufacturing and standard materials have been applied in order to speed up the prototyping, reduce costs and ease the testing and maintenance.
The conception of the structure is based on standard human parameters with the purpose of mimicking the kinematics of human legs and approximating the final outputs as much as possible to the gaits performed by people.
The existing software for simulation and control of bipedal robots in the AI department has been migrated to ROS \cite{ros} and Gazebo \cite{gazebo} to give a bigger portability and generality to the application, and the developed controllers and interfaces are ROS-based as well. 
The electronics hardware utilized in this thesis consists mainly of components developed for the Locokit project \cite{locokit} created at the University of Southern Denmark, which have been adapted to the specific requirements of the task.
Furthermore, in an early stage of the project the platform was meant to be actuated by an existing, although extended, neural control and therefore it has been designed in accordance to its characteristics and constraints.