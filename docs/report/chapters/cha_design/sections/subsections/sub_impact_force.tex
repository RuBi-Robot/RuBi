%!TEX root = ../../../../report.tex
\subsection{Impact force} % (fold)
\label{sub:impact_force}
In order to calculate the physical dimensions of some of the components, the bounding conditions have to be defined.
The presented case shows a peak of energy when landing after being jumped an estimated height.
This energy in then transmitted from the first contact point, the footprint, to the rest of the system, causing stresses that must be absorbed.
The components of the system must receive that energy under a controlled behavior -this is, elastic deformations- assuring a longer service life of the legs.

Thus, some input parameters to calculate the impact force are assumed.
From this force all the consecutive components in the deformation chain will be sized.
Despite the deformation is of the whole system, the security coefficient assumed in here is going to be the calculation of all the components for that maximum force.

From the formula of the mechanical energy:
\begin{equation}
  E_{mechanical} = m g \Delta h + \frac{1}{2} m v^{2}
\end{equation}

The kinetic energy is negligible and only the energy from falling a certain height is supposed.
This energy is then translated into force by supposing a deformation of the whole body as expressed in the equation \ref{eq:impact_force}.
\begin{equation}
\label{eq:impact_force}
  F_{impact} = \frac{m g \Delta h}{t_{impact\_displacement}}
\end{equation}

The equation \ref{eq:impact_force} gives the force for sizing all the components.
Based on the input parameters defined in the appendix \ref{app:profile_selection} which are shown in the table \ref{tab:input_parameter_impact_force}.
\begin{table}
\begin{center}
\begin{tabular}{c | c}
  Parameter & Value \\
  \hline
  Total mass [kg] & 1.5 \\
  Jumping height [m] & 0.1 \\
  Impact displacement [m] & 0.005
\end{tabular}
\caption{Input parameters for calculating the impact force}
\label{tab:input_parameter_impact_force}
\end{center}
\end{table}

The impact force is then:
\begin{equation}
  F_{impact} = \frac{m g \Delta h}{d_{impact\_displacement}} = 294.40 N 
\end{equation}
% subsection impact_force (end)