%!TEX root = ../../../../report.tex
\subsection{Impact force} % (fold)
\label{sub:impact_force}
The assessment of the desired mechanical characteristics of some of the components in the limbs requires the calculation of the internal forces they will be subjected to during the motion.
This process would require the creation of an impact model for the feet and the ground, which due to its complexity has been left as further work.
To approximate the output of this model, the calculations of the energy transferred to a landing leg in collision with the ground have been carried out using the model of falling object hitting the ground.
The assumptions made for the scenario are listed in \ref{list:impact_model}.

\begin{enumerate}
\label{list:impact_model}
	\item No bouncing or slippering 
	\item No losses of energy between states
	\item No compliance on the leg
	\item No deformation on the ground surface
	\item All the potential energy before the fall is dissipated in the impact 
\end{enumerate}

This energy in then transmitted from the first contact point, the footprint, to the rest of the system, causing stresses that must be absorbed.
The components of the system must receive that energy under a controlled behavior -this is, elastic deformations- assuring a longer service life of the legs.
Thus, some input parameters to calculate the impact force are assumed.
From this force all the consecutive components in the deformation chain will be sized.
Despite the deformation is of the whole system, the security coefficient assumed in here is going to be the calculation of all the components for that maximum force.

From the formula of the mechanical energy:
\begin{equation}
  E_{mechanical} = m g \Delta h + \frac{1}{2} m v^{2}
\end{equation}

For a free falling object in the defined scenario, it is assumed that all the potential energy on the first instant is transferred to the crash without intermediary loses.
This energy is then translated into force by supposing a deformation of the whole body as expressed in the equation \ref{eq:impact_force}.

\begin{equation}
\label{eq:impact_force}
  F_{impact} = \frac{m g \Delta h}{d_{impact\_displacement}} = 294.40 N 
\end{equation}

The equation \ref{eq:impact_force} gives the force for sizing all the components.
Based on the input parameters defined in the appendix \ref{app:profile_selection} which are shown in the table \ref{tab:input_parameter_impact_force}.
\begin{table}
\begin{center}
\begin{tabular}{c | c}
  Parameter & Value \\
  \hline
  Total mass [kg] & 1.5 \\
  Jumping height [m] & 0.1 \\
  Impact displacement [m] & 0.005
\end{tabular}
\caption{Input parameters for calculating the impact force}
\label{tab:input_parameter_impact_force}
\end{center}
\end{table}
