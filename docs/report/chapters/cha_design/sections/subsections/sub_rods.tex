%!TEX root = ../../../../report.tex
\subsection{Rods} % (fold)
\label{sub:rods}
As explained in section \ref{sub:bearings}, was decided to have two bearings per link and a rod going though them.
This rod is then also used in the case of the knee and the ankle as a support for the pulleys that transmit the power from the motor to the next link.

Three mechanical efforts bound its design:
\begin{enumerate}
  \item \textbf{Shear strenth}: in the case of the shear produced when and impact occurs and the rod of one link moves in the opposite direcction than its relative in the consecutive link.
  \item \textbf{Resistance to beding}: due to the bending effort that the tension of the belt is constantly applying in the knee and the ankle.
  \item \textbf{Torsion}: due to the pulley in the knee and the ankle. 
  This effort is negligible because zero-friction bearings are supposed.
\end{enumerate}

  \subsubsection{Shear analysis} % (fold)
  \label{ssub:shear_analysis}

  % subsubsection shear_analysis (end)

  \subsubsection{Bending} % (fold)
  \label{ssub:bending}

  % subsubsection bending (end)


% section rods (end)