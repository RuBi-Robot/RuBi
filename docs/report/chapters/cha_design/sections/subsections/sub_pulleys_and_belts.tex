%!TEX root = ../../../../report.tex
\subsection{Pulleys and belts} % (fold)
\label{sub:pulleys_and_belts}

  \begin{enumerate}
    \item Explain why a tranmission system
    \item Different types (belt+pulley, chain, gears..)
    \item Selection of the belt and the pulley
      \item Talk about the different kinds of belt drives and components
      \item Selection of the type G
      \item Calculation of the belt's standar force
  \end{enumerate}

  % 21.220.10: Belt drives and their components (http://www.iso.org/iso/iso_catalogue/catalogue_ics/catalogue_ics_browse.htm?ICS1=21&ICS2=220&ICS3=10)
  
  \subsubsection{Type of the belt system}  % (fold)
  \label{ssub:type_of_the_belt_system}
  
  % subsubsection type_of_the_belt_system (end)

  The best one is the type G in efficiency and reduction of backlash.
  Is appropiate for precission movements, high torques and low speeds.

  Its profile is defined by the ISO 13050:2014 \cite{ISO13050}.

  % subsection belt_and_pulleys_selection (end)

  \subsubsection{Calculation of power rating and drive centre distance} % (fold)
  \label{ssub:calculation_of_power_rating_and_drive_centre_distance}
  ISO 5295:1987 \cite{ISO5295}
  
  % subsubsection calculation_of_power_rating_and_drive_centre_distance (end)

% subsection pulleys_and_belts (end)