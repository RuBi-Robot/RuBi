%!TEX root = ../../../../report.tex
\subsection{Limb profile} % (fold)
\label{sub:limb_profile}
Based on the requirements of weight and its distribution defined in the analysis of the joints \ref{sec:joints}, the links have been decided to have in the upper extreme the motors. 
This leads to use a transmission system, such as belt and pulleys, which leaves for the rest of link an structural function that can also adopt the task of wiring placement.

Thus, a light weight section that satisfy the conditions of deformation and stress maximum will be chosen.
Carbon fiber is an ideal material to achieve this conditions of weight and stress so an quantitative analysis has been made calculating the optimal solution and then rounding it for all the possible profiles offered by the given provider.
The provider was chosen due to the previous experiences that the Mærsk Mc-Kinney Møller Institute had with carbon fiber orders.

The section profile offered \footnote{http://www.easycomposites.co.uk/\#!/cured-carbon-fibre-products/} are: \textit{Rod}, \textit{Tube}, \textit{Box}. The \textit{Stripe} and the \textit{Angle} are discarded due to its asymmetrical geometry that will will lead to less predictable scenarios.

  \subsubsection{subsubsection name} % (fold)
  \label{ssub:subsubsection_name}
  
  % subsubsection subsubsection_name (end)


% subsection limb_profile (end) 