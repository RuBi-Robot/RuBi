%!TEX root = ../../../../report.tex

\subsection{Bearings} % (fold)
\label{sub:bearings}
In the section \ref{sub:impact_force} the force for sizing the bearings of the knee and the ankle was calculated.
The bearing elected would be such that allow dynamic loads of more than the impact force while keeping as small as possible to reduce the added weight to the robot.
On the other hand, the internal dimeter comes defines by the rod diameter calculated in the section \ref{sub:rods}.

An estimation of nominal life of the bearing can be done from the Dynamic Load Rating (C), the Dynamic Equivalent Load (P) and the Life Rime Coefficient for a Ball Bearing (p) (being p=3 for balls bearings). 
\begin{equation}
  L_{10} = \frac{10^{6}}{60 n} \left(\frac{C}{P}\right)^{p} = \frac{10^{6}}{60 n} \left(\frac{403}{294.4}\right)^{3} = 
\end{equation}

The term $L$ is the service life of a bearing (in number of hours or rpm), in normal conditions of speed and load, in which the bearing is working until fail by fatigue. 
Whilst $L_{10}$ is based in a stadistical model that is defined as the 90\% of the bearing of the same type will withstand those loads for a longer time.
% subsection bearings (end)