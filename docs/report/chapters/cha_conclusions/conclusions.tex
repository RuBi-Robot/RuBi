%!TEX root = ../../report.tex
\chapter{Conclusions and further work} % (fold)
\label{cha:conclusions}
As a general conclusion, the creation of a coherent simulation environment and the construction of a robust, low-cost and reconfigurable bipedal locomotion platform has been achieved.
The ultimate goal was to contribute on the on-going line of research in locomotion control at the AI department at the Maersk Mc-Kinney Møller Institute, and RuBi offers a good compromise between features and affordability.
From the bringing up of the robot to the set up of the simulation environment,  including easy manufacturability and maintenance or code examples, RuBi is focused on facilitating the development of new locomotion controllers for teaching or research.
The platform aims to create a bridge between the creation of controllers and their testing, which ultimately will speed up the development workflow and the possible experiments to be carried out in the area by the Embodied AI \& Neurorobotics Lab.

However, much work was also planned to be left for further improvements of the RuBi study framework since, as said before, from the beginning this project was meant to be continued.
This reason has led to the creation of the detailed and comprehensive documentation contained in this report, its appendixes, the assembly manuals \footnote{https://www.youtube.com/watch?v=rceASqIJ4HQ} and the online repository \cite{rubi_repo}.
\todo{any other source?}.
Their aim is to provide the tools to easily and quickly understand and master the use of RuBi.
From here, the uses of the platform are left to the imagination of future users.

\section{Further work} % (fold)
\label{sec:further_work}
The missing work that laid on the original scope of this thesis but could not be conducted is mentioned here for each of the main areas of the project.
Besides, some enhancements and experimental setups already suggested are summarized below.

% section further_work (end)

%%% Mathematical model
% Development of a ground-contact force model --> the current one is a Normal function
The mathematical model took as an assumption the constraint of having the hip and the tip toe remaining over the same vertical line.
By having the exact mechanical properties of the links, as moments of inertia, center of gravity and mass, a model which holds the center of gravity vertically constrained could be obtained.
Furthermore, a the current force model \todo{no tengo ni idea de que es esto nacho, ayuddaaaa} is based on a normal function and could be improved by using a ground-contact force model.

%%% Mechanics
% Analyze all the mechanical stresses of individual parts and an assembly analysis
% Improve fastening for motor interfaces
% Improve foot shape in order to help the step
% Reinforce fragile sections in 3D printed parts
% Develop a method to tension the belts
% Re-implement parallel spring holder to make it easier
% Add more features to the structure
% Implement a jumping platform
From the robot mechanics perspective, the very basic tests conducted seem to yield the need of reinforcing the hip structure, keeping its original design but widening its size.
Also the robot holder on the external structure needs to be 3D-printed again to adjust its clearances due to the use of a different 3D printer for its creation, because of its size.
The fastening of the motor interfaces could also be improved as it has shown to be a valid design but with some leaks when clamping.
Other future improvements could be the development of a more accurate method to measure the tension of the belts or a redesign of the parallel spring holder to make faster the change of the torsion spring.
Besides, in \ref{cha:experiments} it was suggested the construction of a small platform with additional contact sensors for experiments in vertical jumps.
This could be necessary since the current contact sensors are placed on the heels, but the first landing point when hopping is the toes.
Since a solution to the placement of more sensors on the current foot design could not be found, it is proposed to install off-board sensors to supply that information.

%%% Electronics
% Alternatives to the use of the main board's input pins
% Implement a method to read easily from I/O
% Add more sensors to the robot: IMU
% Add absolute encoders or improve the initialization
% Reduce the network latency by using 5Ghz or by using Ethernet
% Study the general latency of the system and how it affect to the measurements
% Develop a easier and safer system for powering the robot
The electronic implementation of RuBi has proved to be robust and valid, but the set of possibilities that offers has not been fully exploited.
The ROS Control interface has to be extended to include position control and readings from new sensors besides the angular encoders in the motors.
The feet contact sensors have been installed but not plugged to the processor due to lack of documentation on the main board.
This could be easily fixed adding some external circuitry as an Arduino, for instance, or exploring the capabilities of the existing I/O pins on the motor controller boards.
Furthermore, abilitating new GPIOS on the hardware system could allow for more and more complex on-board sensors as an IMU or light sensors to improve self-awareness.
Currently the relative encoders on the motors are manually initialized by placing the joints on their mechanical maximum extension.
This should be enhanced to create an automatic solution.

%%% Simulation
% Add springs to the simulation
% Improve the simulation by modifying physic parameters until reach reality-like results
% Add color and textures to the simulation
% Implement more experiments within the simulation
Ultimately, the simulation model of RuBi was designed with direct drive transmission in the joints.
The addition of springs to the model could lead to more realistic results, though this has to be tested.
The adjustment of certain simulation values as the time step or some in the model of the robot, as the dynamic rotational friction, have to be done based on results taken from the real robot.
The model can also be improved by attaching more realistic visual characteristics like adding colors and textures to the parts.
The possibilities of Gazebo have not been squeezed so the inclusion of new simulation environments, sensors or other models is left to further research.

% chapter conclusions (end)