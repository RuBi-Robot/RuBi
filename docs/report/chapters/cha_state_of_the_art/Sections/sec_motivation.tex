%!TEX root = ../../../report.tex
\section{Motivation}
\label{sec_motivation}
Human locomotion arises from the combination of a big variety of subsystems working in conjunction to achieve the intended gait generation according to the requirements of the situation.
The biomechanics of the limbs, consisting in bones, muscles and tendons under the control of the nervous system yields the adequate production of the different motion patterns in order to displace the body as energetically efficiently as possible.
This complex behavior is the result of 4 million years of an evolution \cite{bipedalism} that started in primates and that has entailed both morphological and neurological changes in the human body since the first bipedal hominids to the current structure in homo sapiens. 
Extensive research in the structures involved in human bipedalism has been conducted from within the fields of biology, medicine and sport science almost since their creation. 
But it has been in the last 30 years of the past century when the robotics field has started to focus on accomplishing bipedal locomotion, being the first model the WAM-1, built at Waseda University in 1967.
Since this robot, the evolution of the platforms targeted at mimicking human locomotion has rapidly achieved great successes, which can be exemplified in robots as ASIMO, MAVEL or BioBiped \cite{}.  
However, even though the generation of stable walking gaits for biped platforms seems to have reached a next stage with the latest Atlas robot, the differences in kinematics, kinetics and control required for running and for the transition between the walking and running gaits still set out unsolved challenges.
Besides, differences in energy consumption and balance control or trajectory generation still remain noticeable when comparing the performance of humanoid robots and humans \cite{h7}.

In order to contribute to the study of stable walking and running gaits generation in bipedal robots, and aiming at providing a new tool to gain new insights on the transition between these gaits, this project offers a new robotic platform for continuing the research in this areas at the University of Southern Denmark.
The current bipedal robot being utilized at the AI department at the Mærsk Mc-Kinney Møller Institute to benchmark neural controllers for locomotion is the DACbot walking robot, a next generation of RunBot \cite{runbot1} \cite{runbot2}.
The limitations of this model when trying to approach human-like gait include among others the lack of actuated ankles or compliance in other joints besides the ankles. 
Furthermore, its fixed structure and the way it was manufactured make it hard to modify for different experiments or even repair.
The Rubi robot has been conceived to overcome the above mentioned problems and provide new features that can be of utility in future studies.



Difficulties to calculate compliance added by springs configuration for specific applications.
Possibility to change the springs configurations
Study of adaption of neural controllers to robot platforms whose mathematical model is hard to define due to added compliance.
Possibility of comparing controllers performance in two different platforms
