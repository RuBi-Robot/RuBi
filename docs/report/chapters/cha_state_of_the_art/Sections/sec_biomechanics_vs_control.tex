%!TEX root= ../../../report.tex

\section{Biomechanics meets Control} % (fold)
\label{sec:biomechanics_vs_control}
The first attempts to accomplish artificial walking machines in the mid-1950s were based on stiff structures and kinematic control.
There has been more than 50 years of development in several technology fields between them and the current soft, compliant and torque-controlled legged robots.
Among the main advances that gave rise to the evolution of the autonomous walking robots, three considered essential and of great relevance for this thesis are listened in \ref{list:leg_advances} and further discussed here.
Thanks to them, nowadays 

\begin{itemize}
\label{list:leg_advances}
	\item The realization of the importance that a dynamic and active control has on the walking behavior, as opposed to rigid, kinematics-based motion.
	\item The conception of the embodied AI. The influence of the body in the process of thinking.
	\item The improvements in sensors/actuators performance, material science, computing power and power sources.
\end{itemize}

\subsection{Dynamics of legged locomotion} % (fold)
\label{sub:dynamics_control}
Stiff position control for kinematic trajectory planning has lately demonstrated impressive capabilities like during the DARPA learning locomotion challenge.
However, the limitations of this kind of control such as the need of a very detailed knowledge  about the robot state vector and the environment have been also exposed.
Thus, it seems that the solution for robust and adaptive motion platforms has to go through the development of dynamics-based control models.
The first big steps taken in dynamic legged locomotion control took place in the Leg Laboratory, at the MIT Artificial Intelligence Laboratory by Marc Raibert and his team.
Their findings about the importance of active balance and the possibility of creating simple and generalizable control algorithms for complex dynamics legged systems \cite{mit_leg_lab1} pushed the development of walking robots to a new stage.


% subsection dynamics_control (end)

\subsection{The importance of embodiment in control} % (fold)
\label{sub:the_embodiment_}
%R. Peiffer paper
%Passive walkers
%Compliance
%Inspiration from nature
% subsection the_embodiment_ (end)

\subsection{Main advances in hardware} % (fold)
\label{sub:the_advances_in_hardware}
%First WAP robots bipedalism_history pag 28

% subsection the_advances_in_hardware (end)


% section biomechanics_vs_control (end)