%!TEX root = ../../../report.tex
\section{Robot definition} % (fold)
\label{sec:robot_definition}
As in May of 2016, ROS Jade ecosystem has different formats to describe a robot.
These are SDF, URDF and Xacro.
At some point they are all transformed to a unique sort of format but the decision of how to express the robot for the simulation is needed from the beginning.


\begin{enumerate}
  \item \textbf{URDF \footnote{http://wiki.ros.org/urdf}}: is an open standard used in all the simulators mentioned or others like RobWork \cite{robwork}. 
  It allows to define all the properties of a single robot but lacks other which are important when simulating. 
  It is mainly used for visual representations or schematic robot definitions.
  \item \textbf{Xacro \footnote{http://wiki.ros.org/xacro}}: is a parametric format that easy the writing of the URDF.
  It allows logic conditions like \textit{if, for} and from ROS Jade any virtual python condition.
  This code is then compiled into a URDF automatically which give Xacro complete interoperability with URDF readers.
  \item \textbf{SDF \footnote{http://sdformat.org}}: adapted to the current requirements of the simulation environments.
  With it can be defined from \textit{worlds} to air properties in the case or UAV simulations.
\end{enumerate}

Despite SDF contains more information, there are several tools in ROS Jade that make use of URDF.
Two of them are RViz and ROS Control, explained in the section \ref{sub:ros_control}.
As this last one is a pillar of the framework itself, the decision of making use of URDF as format for describing the robot was taken.
However, the robots developed have been written in Xacro due to the tools that easy the coding of the robot.
% section robot_definition (end)