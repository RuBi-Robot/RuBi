%!TEX root = ../../report.tex
\chapter{Economical analysis} % (fold)
\label{cha:economical_aspects}
In this section the economical aspects of the thesis are detailed.
A core criterion since the initial conception of the RuBi robot and its framework and during its whole development has been its low-cost.
This has reduced the amount of available solutions for the design problems faced, but in exchange it has boosted creativity and out-of-the-box thinking.
This chapter, although short in extension, has been considered relevant enough to be included in the report since the economical constraints have greatly shaped the final prototyping process.

For the calculation of the final, overall budget of this project, the labor cost was neglected due to the difficult to measure the amount of hours dedicated to the project in both sides, the students and the supervisors.
In the same line, the equipment and software costs have not been included because of the uncertainty when calculating the associated cost to the project of, for example, the computers used in its development.
Thus, only the materials cost has been accounted, as shown in Table \ref{tab:material_cost} reaching a total of 22,772 DKK including all the elements used for the project, not only the specifically purchased ones.
Some of the components were available, like the LocoKit, and therefore were not bought.
Others like the tools or the laptops are not added as they are supposed to be already recouped.
Analyzing the budget, the bulk of the money goes into the LocoKit, that includes the electronics and the Maxon motors which are the most expensive part of RuBi.
However the rest of the components show to be affordable what makes its replication and reparation simple from an economical point of view.

\begin{table}[H]
\caption{Material cost.}
\centering
\begin{tabular}{l|r|r|r}
\multicolumn{4}{c}{\Large Material Cost} \\
\multicolumn{1}{c|}{\textbf{Item}} & \multicolumn{1}{c|}{\textbf{Quantity}} & \multicolumn{1}{c|}{\textbf{Unitary cost [DKK]}} & \multicolumn{1}{c}{\textbf{Cost [DKK]}} \\ \hline
20mm(18mm) Carbon Fibre Tube & 1 & 192.51 & 192.51 \\ \hline
10mm(8mm) Carbon Fibre & 1 & 99.21 & 99.21 \\ \hline
8mm(6mm) Carbon Fibre Tube & 1 & 91.97 & 91.97 \\ \hline
Kugleleje SS623-2RS & 26 & 20.67 & 537.42 \\ \hline
Contitech Tandrem & 2 & 137.18 & 274.36 \\ \hline
Låsering & 1 & 28.48 & 28.48 \\ \hline
Rustfrit & 2 & 192.07 & 384.14 \\ \hline
Skrue M3 x 6mm & 1 & 104.82 & 104.82 \\ \hline
Skrue M3 x 3mm & 1 & 96.61 & 96.61 \\ \hline
Stiverprofil & 4 & 91.97 & 367.88 \\ \hline
RS Pro Kugleleje & 2 & 16.44 & 32.88 \\ \hline
SKF Lineært kugleleje & 2 & 116.89 & 233.78 \\ \hline
Fastgørelse og forbindelsesled & 1 & 62.76 & 62.76 \\ \hline
RS Pro Glat bøsning OB368 & 1 & 54.06 & 54.06 \\ \hline
Skrue M6 x 10mm & 1 & 34.61 & 34.61 \\ \hline
Torsion springs & 1 & 1,454 & 1,454 \\ \hline
3D printer filament & 1 & 140.00 & 140.00 \\ \hline
LocoKit  & 1 & 18,592.00 & 18,592.00 \\ \hline
\multicolumn{2}{l}{} & \textbf{Total [DKK]} & \textbf{22,772.19} \\
\end{tabular}
\label{tab:material_cost}
\end{table}


% chapter economical_aspects (end)